\documentclass[12 pt]{article}  % sets the font to 12 pt and says this is an 
                                % article (as opposed to book or other documents)
\usepackage{amsfonts, amssymb}  % math fonts and symbols
\usepackage{amsmath}            % math
\usepackage{enumitem}           % change enumeration bullet styles
\usepackage{graphicx}           % graphics
\usepackage{xcolor}             % color
\usepackage{hyperref}           % linkable references
\hypersetup{
    colorlinks=true,
    linkcolor=blue,
    filecolor=magenta,      
    urlcolor=cyan
    }
  
%\usepackage{setspace}          % Together with \doublespacing below allows for
                                % doublespacing of the document

\oddsidemargin=-0.5cm
\setlength{\textwidth}{7in}
\addtolength{\voffset}{-20pt}
\addtolength{\headsep}{25pt}
\renewcommand\thefootnote{[\arabic{footnote}]} % Bracketed footnotes

                                        % Macro definitions for the header
\newcommand{\myteam}{Miles Leonard,
                     Danny Nguyen}      % Name
\newcommand{\myid}{OU ID: 113687989}    % Student ID
\newcommand{\myclass}{DSA 5103}         % Class
\newcommand{\myassignment}{Project Proposal}  % Assignment designator

% Page headers automatically horizontal fill; Name ID Date Class Assignment Page.No.
\pagestyle{myheadings}
\markright{\myteam \hfill \today \hfill \myclass \hfill \myassignment \hfill }

\newcommand{\eqn}[0]{\begin{array}{rcl}}    % begin an aligned equation - allows
                                            % for aligning = or inequalities.
                                            % Always use with $$ $$
\newcommand{\eqnend}[0]{\end{array} }  	    % end the aligned equation

\newcommand{\qed}[0]{$\square$}     % make an unfilled square the
                                    % default for ending a proof

% Common number sets, must be in math mode
\newcommand{\Naturals}[0]{\mathbb{N}}   % Natural numbers
\newcommand{\Integers}[0]{\mathbb{Z}}   % Integers
\newcommand{\Rationals}[0]{\mathbb{Q}}  % Rationals
\newcommand{\Reals}[0]{\mathbb{R}}      % Reals
\newcommand{\Complex}[0]{\mathbb{C}}    % Complex

\newcommand{\heading}[1]{\noindent\large{\textbf{#1}}}  % for sections
\newcommand{\problem}[1]{\noindent\paragraph{#1:}}      % for problem labels

%\doublespacing         % Together with the package setspace above allows for
                        % doublespacing of the document

% Use these commands for self-assessments
\newcommand{\correct}[1]{\colorbox{green}{\checkmark #1}}   % mark correct
\newcommand{\correction}[1]{{\color{red}\textbf{#1}}}       % write a correction

\begin{document}

\section*{Project Proposal}

\paragraph{Team Members:} Miles Leonard, Danny Nguyen

\paragraph{Project Title:} Modeling light pollution using various geospatial data sources

\paragraph{Description of Data/Problem:} Light pollution is a night-time phenomenon
caused by the atmospheric scattering of excess light from things such as street 
lamps or other outdoor light fixtures that then bounces off of molecules in the air. 
Light pollution is commonly measured as night sky brightness (in magnitudes per square 
arcsecond), but is typically only directly measured at sites like observatories. 
Even then, data from these observatories is rarely accessible in real-time. The 
Globe at Night campaign created a monitoring network for observatories across the 
world to provide night sky brightness data for public use, with data going back 
as far as 2014. Light pollution, being an atmospheric phenomenon, is heavily 
impacted by weather conditions. The objective for this project is to predict light 
pollution (night sky brightness) at a given location using primarily meteorological, 
energy usage, and population data.

\paragraph{Proposed Solution:} Using supervised learning techniques, associate 
the light pollution data for observatories to their distances from population 
centers, local energy usage, and meteorological conditions. We plan to use the 
processed data to construct a regression model to predict light pollution for 
any location.

\paragraph{Data Sources:}

\begin{enumerate}
    \item Light pollution data (night sky brightness)\\
    \url{https://globeatnight.org/gan-mn/}

    \item Meteorological data (multiple variables):\\
    \url{https://cds.climate.copernicus.eu/datasets/reanalysis-era5-pressure-levels}

    \item U.S. Energy Information Administration Open Data (Maybe draw a correlation to state energy usage?)\\
    \url{https://www.eia.gov/opendata/}

    \item International Energy Agency (country-level data on energy topics)\\
    \url{https://www.iea.org/data-and-statistics/data-sets?filter=free}

    \item US Population Data\\
    \url{https://www.census.gov/data/datasets.html}

    \item Population Weighted Centroid Positions by Zip Code (US only)\\
    \url{https://catalog.data.gov/dataset/zip-code-population-weighted-centroids}

    \item Census International Database Subnational Data\\
    \url{https://www.census.gov/data-tools/demo/data/idb/excel/IDBSubnational.zip}

\end{enumerate}

\end{document}
